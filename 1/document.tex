\documentclass{article}

\usepackage[T1,T2A]{fontenc}
\usepackage[utf8]{inputenc}
\usepackage[russian]{babel}
\usepackage{ dsfont }
\usepackage{mathtools}
\usepackage{ gensymb }
\usepackage{ amssymb }
\usepackage[left=1cm,right=1cm,
top=1cm,bottom=2cm,bindingoffset=0cm]{geometry}
\begin{document}
\large
	Пак Николай, 1ДЗ, 696\\
	\begin{enumerate}
		\item 
			\begin{enumerate}
				\item 
				\item
					$b^*(a^* + c^* + c^+b^*)^*$\\
					Так как в условии требуется чтобы после a не шло b, то перед b может идти только c или b, тогда $(a^* + c^* + c^+b^*)^*$ задает, требуемое множество слов где слова не начинаются с буквы b, а добавка $b^*$ в начало дополняет до всех случаев.  
						
			\end{enumerate}
		\item 
			Если p ограничено, то и n ограничено, значит все $a^n$ можно перечислить. Иначе все $n \in N$ подходят, занчит можно записать данное множество как $a^*$
		\item
			Пусть $w \in L(e)$, тогда 
			$\left[ 
			\begin{gathered} 
			 w \in L(g)\\
			 w = w'u, w' \in L(e), u \in L(f)
			\end{gathered}  \Rightarrow
			\left\{ 
		\begin{gathered} 
		w \in L(g)\\
		w = w'u, w' \in L(e), u \in L(f)
		\end{gathered} 
			
						\right.$\\\\
			Если первое, то $w \in gf^*$\\ 
			Иначе оно заканчивается на слово из f. Теперь рассмотрим w', аналогично w и так далее, так как слово конечно, то на последнем шаге если получится $\varepsilon$, то если в L(e) есть пустое слово, то оно есть в L(g), т.к в L(f) нет пустого слова, иначе если пустого слова нет в L(e), тогда этот случай невозможен. Т.о $w \in gf^*$ 	  
		\item
			$b^*(a + (ab^+)^*)^*$\\
			Данное множество описывает всевозможные комбинации из букв a и b, так как если слово начинается с букв b(не начинается), то $b^*$ это позволяет, дальше $(a + (ab^+)^*)^*$ описывет всевозможные комбинации из a и b, начинающиеся с a, так как оно обеспечивает конструкции состоящие из последоватьности букв а и последовательности из b, перед которой есть одна буква а, и их комбинации.
		\item
			Рассмотрим $(a+b)^R$. $(a+b)^R = (a^R + b^R)$, так как если развернуть все слова во множествах и их объединить, то же самое, что и сначала объединить, а потом развернуть.\\
			Аналогично $(ab)^R = (b^Ra^R)$\\
			А так как $(a^*)^R = (a^R)^*$ (и для $^+$ тоже)(по доказанному на семинаре) и т.к у язака L есть регулярное выражение, его описывающее, то достаточно применить к нему $^R$, чтобы получить $L^R$
	\end{enumerate}
 
\end{document}