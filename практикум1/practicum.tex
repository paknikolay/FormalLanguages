\documentclass{article}

\usepackage[T1,T2A]{fontenc}
\usepackage[utf8]{inputenc}
\usepackage[russian]{babel}
\usepackage{ dsfont }
\usepackage{mathtools}
\usepackage{ gensymb }
\usepackage{ amssymb }
\usepackage[left=1cm,right=1cm,
top=1cm,bottom=2cm,bindingoffset=0cm]{geometry}
\begin{document}
	\LARGE
	\begin{flushleft}
		Пак Николай, 696, практикум \\
		13)  Даны $\alpha$ и слово $u \in {a,b,c}^*$. Найти длину самого длинного префикса $u$, являюегося также префиксом некоторого слова в $L$.\\
		
		
		Будем находить ответ по индукции по длине регулярного выражения. Будем хранить матрицы $deducibleSubsequencesFrefixes$ и $deducibleSubsequences$ размером  $(|L| + 1) \times (|L| + 1)$ для каждого регулярного выражения, поддерживая инвариант :\\ $deducibleSubsequencesFrefixes[i][j] = true \Leftrightarrow$ в соответствующем регулярном выражении есть слово, префикс которого совпадает с подстрокой слова с $i$-й до $i+j$-й позиции,\\
		$deducibleSubsequences[i][j] = true \Leftrightarrow$ в соответствующем регулярном выражении есть слово, которого совпадает с подстрокой слова с $i$-й до $i+j$-й позиции.\\
		$i$ - позиция, $j$ - длина слова.\\ 
		База:\\
		Для букв, найдем ее вхождение в слово и проставим $true$ в соответствующее $deducibleSubsequencesFrefixes[i][1]$ и $deducibleSubsequences[i][1]$, для пустого слова проставим $true$ в соответствующие нулевые позиции(0-я длина). \\
		Из построения видно, что инвариант выполнен.
		
		Переход:\\

		Первое регулярное выражение $a$, второе $b$, результат $c$\\
		\begin{enumerate}
			\item
				Для операции $"+"$ - побитовое или в соответствующих позициях.\\
				Операция $"+"$ - объединение выводимых префиксов и подслов, поэтому инвариант при этом действии сохранится.
				
			\item
				Для операции $"."$ - для каждого $a.deducibleSubsequences[i][j] = true$ первого регулярного выражения перебираем длину выводимого подслова в  $b.deducibleSubsequences$, начинаеющегося в позиции $i+j$, второго регулярного выражения и если нашлась такая позиция, то ставим $true$ в $c.deducibleSubsequences[i][j+k]$, где $k$ - длина найденого подслова во втором регулярном выражении. Т.е. для каждого подслова из первого регулярного выражения продлеваем его подсловом из второго регулярного выражения.\\
				Значения $c.deducibleSubsequencesFrefixes[i][j]$ установим занчениям $c.deducibleSubsequences$. Однако для префиксов нужно добавить все префиксы из первого регулярного выражения и продлить выводимые подслова префиксами из второго регулярного выражения. т.е добавляем
				$a.deducibleSubsequencesFrefixes[i][j]$ первого регулярного выражении и  проделываем аналогичную операцию, описанную выше,  только перебираем в  $b.deducibleSubsequencesFrefixes$.\\
				Из построение $c.deducibleSubsequences$ и $c.deducibleSubsequencesFrefixes$ видно, что если в $[i][j]$-й позиции стоит $true$, то оно получилось при конкатенации выводимых подслов из $a$ и $b$, а в префиках добавляется возможные выводимые префиксы в $a$ и продолжения выводимых слов в $a$ префиксами $b$, поэтому и этом случае инвариант сохраняется. //
				
			\item
				Для операции $"^*"$, присвоим $c.deducibleSubsequences[i][j+k]$ значения $a.deducibleSubsequences[i][j+k]$ и проставим в позиции $[i][0] true$ т.к выводится пустое слово, для каждого (появившегося в процессе) $c.deducibleSubsequences[i][j+k] = true$ продливаем его подсловом в $c.deducibleSubsequences[i][j+k]$, тогда получится итеративное приписывание выводимых подслов, т.е получится замыкание Клини.\\
				Значениям $c.deducibleSubsequencesFrefixes[i][j]$ присваиваем занчения $c.deducibleSubsequences[i][j+k]$.\\
				Осталось добавить возможные префиксы к получившимся подсловам: для $c.deducibleSubsequencesFrefixes[i][j] = true$ продлеваем его префиксом из $a.deducibleSubsequencesFrefixes$, и добавляем занчения $deducibleSubsequencesFrefixes$\\
				Из построения видно, что значение $true$ появляется в $c.deducibleSubsequencesFrefixes$ и $c.deducibleSubsequences$, когда можно было вывести на каком-то шаге из двух выводимых подслов меньших длин, которые в свою очередь были выведены из более меньших. Т.о. они были выводились из выводимых слов в $a$ замыканием Клини, значит инвариант сохранится.\\
			\end{enumerate} 
		Для ответа нужно найти максимальное значение $j$ при условии  $c.deducibleSubsequences[0][j] = true , j \in {0,1,2, \ldots |L|}$. Т.е. находим максимум среди  всевозможных выводимых префиксов слова в $c$ - конечном регулярном выражении.
	\end{flushleft}
\end{document}